\section{Συμπεράσματα-Μελλοντική Εργασία}

Στην εργασία αυτή, έγινε μία θεωρητική ανασκόπηση του πώς μη-μεταθετικές ομάδες και κυρίως η ομάδα των Braids μπορούν να χρησιμοποιηθούν για να υλοποιήσουν κρυπτογραφικά πρωτόκολλα. Επιπλέον παρουσιάστηκαν  δύο τέτοια πρωτόκολλα καθώς και οι τρόποι επίθεσης σε αυτά.

Έπειτα παρουσιάστηκε μία υλοποίηση για τα δύο αυτά κρυπτοσυστήματα με την χρήση του Matlab. 
Τα δύο κύρια σημεία της υλοποίησης αφορούν την μετατροπή από και προς κανονική μορφή και την επίθεση.

Αρχικά τροποποιήσαμε τον αλγόριθμο της κανονικής μορφής και τον υλοποιήσαμε. Όπως αναφέραμε για να γίνει αποδοτικότερη η υλοποίηση αλλά και για να υπάρχει μοναδικός τρόπος αναπαράστασης ενός στοιχείου περιοριστήκαμε στην $ B_4 $. Σαν μελλοντική υλοποίηση θα μπορούσε να δοκιμαστεί το σύστημα για διαφορετικά $ n $, ίσως ακόμα και να υλοποιηθεί για μεταβλητό $ n $, αν και δεν υπάρχει προφανής τρόπος με τον οποίο ένα permutation να μπορεί μοναδικά να αντιστοιχηθεί σε κάποιο στοιχείο του $ B_n $.

Όσον αφορά την επίθεση, ο αλγόριθμος που υλοποιήθηκε σπάει τα κρυπτοσυστήματα περίπου τρεις φορές γρηγορότερα από την εξαντλητική επίθεση. Και γενικότερα αυτός ο αλγόριθμος, είναι πολυωνυμικού χρόνου, όμως, για λόγους που αναπτύχθηκαν,  δεν επιτυγχάνει πάντα. Στο σύστημα μας το ποσοστό επιτυχίας ήταν περίπου 1/4. Σαν μελλοντική εργασία, το ποσοστό αυτό θα μπορούσε να αυξηθεί, εντάσσοντας στους κώδικες περισσότερους ελέγχους ως προς την καταλληλότητα του κλειδιού που επιστρέφεται. 
Προφανείς βελτιώσεις είναι να πιστοποιείται να ελέγχεται ότι το κλειδί που επιστρέφεται ικανοποιεί τα κριτήρια που μπορεί να ελέγξει η Eve, όπως για παράδειγμα ότι έχει το σωστό μήκος ή αποτελείται από στοιχεία που ανήκουν στην κατάλληλη υποομάδα του $ B_n $. Άλλη πρόταση είναι να επιστρέφονται πολλά κλειδιά και να ελέγχεται η εγκυρότητα όλων. Όμως χρειάζεται μελέτη για το ποιοι από τους ελέγχους αυτός αξίζει να προστεθούν, δηλαδή εισάγουν επιπλέον κόστος που ισοσταθμίζεται από ανάλογη μείωση της πιθανότητας αποτυχίας.

Γενικά, από τις παρατηρήσεις προκύπτει ότι η πιθανότητα επιτυχίας αλγορίθμου είναι $1/poly(n)$, αν το μήκος των ιδιωτικών κλειδιών είναι πολυωνυμικό ως προς το μήκος της ανα$ n $.  Ένα τέτοιο ποσοστό, αν και είναι μικρό, στην κρυπτογραφία θεωρείται μη αμελητέο.
Επομένως αυτά τα κρυπτοσυστήματα δεν είναι ασφαλή σε αυτή την μορφή τους, και είναι ανοιχτό πρόβλημα το αν μπορούν να τροποποιηθούν έτσι ώστε να αποδεικνύεται η ασφάλειά τους. Επίσης είναι ανοιχτό πρόβλημα η ασφάλειά τους, όταν χρησιμοποιούν άλλες μη μεταθετικές ομάδες, όπως αυτές που αναφέραμε στο θεωρητικό κομμάτι της εργασίας.
